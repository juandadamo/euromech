\documentclass[10pt,fleqn,a4paper,twoside]{article}
\usepackage[T1]{fontenc}
\usepackage[utf8]{inputenc}
\usepackage{graphicx,subcaption,adjustbox}
\usepackage{lmodern,tikz,pgfplots,amsmath,siunitx,amssymb}
\usetikzlibrary{arrows,positioning}
\usetikzlibrary{plotmarks}
\usetikzlibrary{3d,calc}
\usetikzlibrary{external,patterns,arrows.meta}
\tikzexternalize
\usepackage{pgfplotstable}
%\usepgfplotslibrary{groupplots}
%\usepackage[top=1cm, bottom=2cm, left=2.3cm,right=2.3cm]{geometry}
\usepackage{babel,amsmath}
%\usepackage[numbers]{natbib}
\usepackage{adjustbox}
\pgfplotsset{compat=newest} 
%opening
\title{}
\author{}
\pgfplotsset{scaled x ticks=false}
\graphicspath{{pdfs}}
\begin{document}
	
	\maketitle
	
	\begin{abstract}
		
	\end{abstract}
	
	\section{}
Results for a model $D = 20mm$.
	\begin{figure}
		
		\centering
		
		\begin{subfigure}[t]{.49\columnwidth} \begin{adjustbox}{varwidth=\textwidth,fbox,center}	
				\resizebox{\columnwidth}{!}{%	
					\tikzsetnextfilename{tikzs/CD_B0.45}			
					\input{tikzs/CD_B0.45.tikz}
				}	\centering				
				\caption{$B=0.45\cdot10^{-5}\SI{}{\newton\meter}$.}\label{fig:B004}	\end{adjustbox}
		\end{subfigure}
		\begin{subfigure}[t]{.49\columnwidth}	\begin{adjustbox}{varwidth=\textwidth,fbox,center}
				\resizebox{\columnwidth}{!}{%	
					\tikzsetnextfilename{tikzs/CD_B5.59}			
					\input{tikzs/CD_B5.59.tikz}
				}						
				\caption{$B=5.58\cdot10^{-5}\SI{}{\newton\meter}$.}\label{fig:B050}	\end{adjustbox}
		\end{subfigure}
		\vfill
		\begin{subfigure}[t]{.49	\columnwidth}	\begin{adjustbox}{varwidth=\textwidth,fbox,center}
				\resizebox{\columnwidth}{!}{%	
					\tikzsetnextfilename{tikzs/CD_B45.44}			
					\input{tikzs/CD_B45.44.tikz}
				}						
				\caption{$B=45.1\cdot10^{-5}\SI{}{\newton\meter}$.}\label{fig:B454}	\end{adjustbox}
		\end{subfigure}
			\caption{Drag coefficient obtained for three flexural rigidities in function of Reynolds number. The varyng parameter for each plot is the length of the flexible flaps $\ell/D$ attached to the D-shaped cylinder.}\label{fig:cd}
		\end{figure} 
%	\begin{figure}		
%			\begin{subfigure}[t]{\textwidth}
%			\centering\resizebox{\textwidth}{!}{%
%\includegraphics[width=\textwidth]{pdfs/CD_RE_50}	
%			} \subcaption{}
%		\end{subfigure}
%	
%	\end{figure}
	\begin{figure}		
	\begin{subfigure}[t]{.5\textwidth}
		\centering\resizebox{\textwidth}{!}{%
				\tikzsetnextfilename{pdfs/CD_RE_50}	
		\input{tikzs/CD_RE_50.tikz}	
		} 	\subcaption{$B=4.43\cdot10^{-5}\SI{}{\newton\meter}$.}\label{fig:B043}
	\end{subfigure}
	\begin{subfigure}[t]{.5\textwidth}
	\centering\resizebox{\textwidth}{!}{%
		\tikzsetnextfilename{pdfs/CD_RE_75}	
		\input{tikzs/CD_RE_75.tikz}	
	} \subcaption{$B=23.4\cdot10^{-5}\SI{}{\newton\meter}$.}\label{fig:B234}
\end{subfigure}	


\caption{Results for a model $D = 50mm$.}

\end{figure}
Flexibility can reduce drag in these cases.

In experiences with $D=50mm$, two elastic modes can be found. Mode 1 \textit{small} lengths. 
We found, for $B=4.43\cdot10^{-5}\SI{}{\newton\meter}$ Mode 1 in $\ell/D$=0.75. For $B=23.4\cdot10^{-5}\SI{}{\newton\meter}$
 $\ell/D\le 1.00$. 
 
 It is important in order to define a reduce velocity number $U_R = f_n D / u_\infty$
 
 Defining a Cauchy number
 $$Ca = \frac{\rho u_\infty^2\ell^3}{8B}$$
 
 
 
 
\end{document}
